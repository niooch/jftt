\documentclass[11pt,a4paper]{article}

\usepackage[T1]{fontenc}
\usepackage[utf8]{inputenc}
\usepackage[polish]{babel}
\usepackage{lmodern}
\usepackage[final]{microtype}

\usepackage[a4paper,margin=2.5cm]{geometry}
\usepackage{fancyhdr}
\pagestyle{fancy}
\fancyhf{} % czyść wszystko
\rhead{\leftmark}
\cfoot{\thepage}

\usepackage{amsmath,amssymb,amsthm,mathtools}
\usepackage{siunitx} % jeżeli potrzebne jednostki
\sisetup{locale = DE} % separator 1,23 (polski), można zmienić na locale=PL gdy dostępne
\usepackage{bm}       % pogrubione symbole

\usepackage[hidelinks]{hyperref}
\usepackage[nameinlink,capitalise,noabbrev]{cleveref}
\usepackage{csquotes}

\usepackage{enumitem}
\setlist{noitemsep,topsep=3pt}
\usepackage{xcolor}

\numberwithin{equation}{section}
\setcounter{secnumdepth}{3}
\setcounter{tocdepth}{2}

% --- Meta ---
\title{Zadanie 5. -- Lista 1.}
\author{Jakub Kogut}
\date{\today}

\begin{document}
\maketitle
\section{Wprowadzenie}
Do rozwiązania zadania będę wykorzystywał następujące definicje i twierdzenia podane na wykładzie.\\
Niech $L$ $L_1$ i $L_2$ będą językami nad alfabetem $\Sigma$. Wtedy:
\begin{equation}
    L_1L_2 = \left{ xy: x \in L_1, y \in L_2 \right}
\end{equation}
\begin{equation}
    L^0 = \left{ \varepsilon \right}
\end{equation}
\begin{equation}
    L^{i+1} = LL^i \quad \text{dla } i > 0
\end{equation}
\begin{equation}
    L^* = \bigcup_{i=0}^{\infty} L^i
\end{equation}
Jeżeli $r$ i $s$ są wyrażeniami regularnymi reoprezentującymi języki $L(r)$ i $L(s)$, to:
\begin{itemize}
    \item $r + s$ jest wyrażeniem regularnym reprezentującym język $L(r) \cup L(s)$,
    \item $rs$ jest wyrażeniem regularnym reprezentującym język $L(r)L(s)$,
    \item $r^*$ jest wyrażeniem regularnym reprezentującym język $L(r)^*$.
\end{itemize}

\section{Zadanie 5.}
Należy udowodnić następujące tożsamości dla wyrażeń regularnych $r$, $s$ i $t$, przy czym $r=s$ oznacza, że $L(r) = L(s)$.
\begin{enumerate}
    \item
        \begin{equation}
            \begin{aligned}
                (r+s)+t &= (L(r)\cup L(s))\cup L(t) \\
                        &= L(r)\cup\bigl(L(s)\cup L(t)\bigr) \\
                        &= r + (s+t)
            \end{aligned}
        \end{equation}
    \item
        \begin{equation}
            \begin{aligned}
                (rs)t &= (L(r)L(s))L(t) \\
                        &= \left\{ xy: x \in L(r)L(s), y \in L(t) \right\} \\
                        &= \left\{ xy: \left( x \in L(r), y \in L(s)\right), y \in L(t) \right\} \\
                        &= \left\{ xy: x \in L(r), y \in L(s)L(t) \right\} \\
                        &= L(r)(L(s)L(t)) \\
                        &= r(st)
            \end{aligned}
        \end{equation}
    \item
        \begin{equation}
            \begin{aligned}
                (r+s) t &= (L(r)\cup L(s))\,L(t) \\
                        &= \left\{ xy: x \in L(r)\cup L(s), y \in L(t) \right\} \\
                        &= \left\{ xy: \left( x \in L(r), y \in L(t)\right) \wedge \left( x \in L(s), y \in L(t)\right) \right\} \\
                        &= \left\{ xy: x \in L(r), y \in L(t)\right\} \cup \left{ xy: x \in L(s), y \in L(t)\right\} \\
                        &= L(r)L(t) \cup L(s)L(t) \\
                        &= rt + st
            \end{aligned}
        \end{equation}
    \item
        \begin{equation}
            \begin{aligned}
                r(s +t ) &= L(r)(L(s)\cup L(t)) \\
                        &= \left\{ xy: x \in L(r), y \in L(s)\cup L(t) \right\} \\
                        &= \left\{ xy: \left( x \in L(r), y \in L(s)\right) \wedge \left( x \in L(r), y \in L(t)\right) \right\} \\
                        &= \left\{ xy: x \in L(r), y \in L(s)\right\} \cup \left{ xy: x \in L(r), y \in L(t)\right\} \\
                        &= L(r)L(s) \cup L(r)L(t) \\
                        &= rs + rt
            \end{aligned}
        \end{equation}
    \item Kożystając z definicji otoczki Kleenego:
        \begin{equation}
            \begin{aligned}
                \emptyset^* &= \bigcup_{i=0}^{\infty} \emptyset^i \\
                        &= \emptyset^0 \cup \bigcup_{i=1}^{\infty} \emptyset^i \\
                        &= \left\{ \varepsilon \right\} \cup \emptyset \\
                        &= \left\{ \varepsilon \right\} \\
                        &= \varepsilon
            \end{aligned}
        \end{equation}
    \item
        \begin{equation}
            \begin{aligned}
                (r^*)^* &= \left( \bigcup_{i=0}^{\infty} L(r)^i \right)^* \\
                        &= \bigcup_{j=0}^{\infty} \left( \bigcup_{i=0}^{\infty} L(r)^i \right)^j \\
                        &= \bigcup_{j=0}^{\infty} \left( L(r)^0 \cup L(r)^1 \cup L(r)^2 \cup \ldots \right)^j \\
                        &= \bigcup_{j=0}^{\infty} \left( L(r)^0 L(r)^0 \ldots L(r)^0 \cup L(r)^1 L(r)^0 \ldots L(r)^0 \cup \ldots \right. \\
                        &\quad \left. \cup L(r)^i L(r)^k \ldots L(r)^m \cup \ldots \right) \\
                        &= \bigcup_{j=0}^{\infty} L(r)^0 \cup L(r)^1 \cup L(r)^2 \cup \ldots \\
                        &= r^*
            \end{aligned}
        \end{equation}
    \item 
        \begin{equation}
            \begin{aligned}
                r^*s^* &= \left( \bigcup_{i=0}^{\infty} L(r)^i \right) \left( \bigcup_{j=0}^{\infty} L(s)^j \right) \\
                        &= \bigcup_{i=0}^{\infty} \bigcup_{j=0}^{\infty} L(r)^i L(s)^j \\
                        &= \bigcup_{k=0}^{\infty} \left( L(r)^0 L(s)^k \cup L(r)^1 L(s)^{k-1} \cup \ldots \cup L(r)^k L(s)^0 \right) \\
                        &= (r+s)^*
            \end{aligned}
        \end{equation}
\end{enumerate}
\end{document}

