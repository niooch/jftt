% !TeX program = pdflatex
\documentclass[11pt,a4paper]{article}

% --- Język i czcionki ---
\usepackage[T1]{fontenc}
\usepackage[utf8]{inputenc}
\usepackage[polish]{babel}
\usepackage{lmodern}
\usepackage[final]{microtype}

% --- Układ strony i nagłówki/stopki ---
\usepackage[a4paper,margin=2.5cm]{geometry}
\usepackage{fancyhdr}
\pagestyle{fancy}
\fancyhf{} % czyść wszystko
\lhead{Notatki z matematyki}
\rhead{\leftmark}
\cfoot{\thepage}

% --- Matematyka ---
\usepackage{amsmath,amssymb,amsthm,mathtools}
\usepackage{siunitx} % jeżeli potrzebne jednostki
\sisetup{locale = PL} % separator 1,23 (polski), można zmienić na locale=PL gdy dostępne
\usepackage{bm}       % pogrubione symbole

% --- Odsyłacze i tytuły ---
\usepackage[hidelinks]{hyperref}
\usepackage[nameinlink,capitalise,noabbrev]{cleveref}
\usepackage{csquotes}

% --- Listy i drobiazgi ---
\usepackage{enumitem}
\setlist{noitemsep,topsep=3pt}
\usepackage{xcolor}

% --- Środowiska twierdzeń po polsku ---
\newtheoremstyle{polski}% name
  {6pt}{6pt}% space above/below
  {\itshape}% body font
  {}% indent
  {\bfseries}% head font
  {.}% punctuation
  {0.5em}% spacing after head
  {}% head spec
\theoremstyle{polski}
\newtheorem{tw}{Twierdzenie}[section]
\newtheorem{lem}[tw]{Lemat}
\newtheorem{wniosek}[tw]{Wniosek}

\newtheoremstyle{polski-def}
  {6pt}{6pt}{\normalfont}{}{\bfseries}{.}{0.5em}{}
\theoremstyle{polski-def}
\newtheorem{defi}[tw]{Definicja}
\newtheorem{przyklad}[tw]{Przykład}
\newtheorem{uwaga}[tw]{Uwaga}

% --- Nazwy do cleveref po polsku ---
\crefname{tw}{twierdzenie}{twierdzenia}
\Crefname{tw}{Twierdzenie}{Twierdzenia}
\crefname{lem}{lemat}{lematy}
\Crefname{lem}{Lemat}{Lematy}
\crefname{wniosek}{wniosek}{wnioski}
\Crefname{wniosek}{Wniosek}{Wnioski}
\crefname{defi}{definicja}{definicje}
\Crefname{defi}{Definicja}{Definicje}
\crefname{przyklad}{przykład}{przykłady}
\Crefname{przyklad}{Przykład}{Przykłady}
\crefname{uwaga}{uwaga}{uwagi}
\Crefname{uwaga}{Uwaga}{Uwagi}
\crefname{equation}{równanie}{równania}
\Crefname{equation}{Równanie}{Równania}
\crefname{section}{sekcja}{sekcje}
\Crefname{section}{Sekcja}{Sekcje}

% --- Szybkie skróty ---
\newcommand{\N}{\mathbb{N}}
\newcommand{\Z}{\mathbb{Z}}
\newcommand{\Q}{\mathbb{Q}}
\newcommand{\R}{\mathbb{R}}
\newcommand{\C}{\mathbb{C}}
\DeclareMathOperator{\sgn}{sgn}
\DeclareMathOperator{\Var}{Var}
\DeclareMathOperator{\E}{\mathbb{E}}

% --- Numeracja i wygląd sekcji ---
\numberwithin{equation}{section}
\setcounter{secnumdepth}{3}
\setcounter{tocdepth}{2}

% --- Meta ---
\title{Gramatyka listy 3.}
\author{Jakub Kogut}
\date{\today}

\begin{document}
\maketitle
Na liscie 3. pojawia się konieczność zkonstruowania gramatyki bezkontekstowej aby móc napisać parser dla translatora wyrażeń arytmetycznych w ciele $GF(1234577)$ z postaci infinksowej do postaci postfiksowej \texit{(RPN)}.
\section{Gramatyka bezkontekstowa}
Niech $G$ -- gramatyka bezkontekstowa zdefiniowana jako $G = (N, T, P, S)$ gdzie:
\begin{itemize}
    \item $N = \{input, line, expr, sum, prod, unary, power, atom, EOF, EOF\}$
        \begin{itemize}
            \item EOL -- znak końca linii
            \item EOF -- znak końca pliku
        \end{itemize}
    \item $T = \{NUM, PLUS, MINUS, MUL, DIV, POW, LPAR, RPAR\}$, gdzie:
        \begin{itemize}
            \item $NUM$ jest ciągiem cyfr reprezentujących liczby w ciele $GF(1234577)$, \texttt{[0-9]+}
            \item $PLUS$ -- znak dodawania +
            \item $MINUS$ -- znak odejmowania -
            \item $MUL$ -- znak mnożenia *
            \item $DIV$ -- znak dzielenia /
            \item $POW$ -- znak potęgowania $\wedge$
            \item $LPAR$ -- lewy nawias (
            \item $RPAR$ -- prawy nawias )
        \end{itemize}
        tyle nieterminali jest konieczne ze względu na priorytety działań oraz konieczność obsługi liczb ujemnych (\textit{unary}).
    \item $P$ -- zbiór produkcji:
    \begin{itemize}
        \item $input \to line\ EOF$
        \item $line \to expr\ EOL$
        \item $expr \to sum$
        \item $sum \to sum\ PLUS\ prod$ | $sum\ MINUS\ prod$ | $prod$
        \item $prod \to prod\ MUL\ unary$ | $prod\ DIV\ unary$ | $unary$
        \item $unary \to MINUS\ power$ | $power$
        \item $power \to atom\ POW\ unary$ | $atom$
        \item $atom \to NUM$ | $LPAR\ expr\ RPAR$
    \end{itemize}
    \item $S = input$
\end{itemize}

\end{document}

