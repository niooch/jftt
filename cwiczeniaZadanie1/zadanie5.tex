\documentclass[11pt,a4paper]{article}

\usepackage[T1]{fontenc}
\usepackage[utf8]{inputenc}
\usepackage[polish]{babel}
\usepackage{lmodern}
\usepackage[final]{microtype}

\usepackage[a4paper,margin=2.5cm]{geometry}
\usepackage{fancyhdr}
\pagestyle{fancy}
\fancyhf{} % czyść wszystko
\rhead{\leftmark}
\cfoot{\thepage}

\usepackage{amsmath,amssymb,amsthm,mathtools}
\usepackage{siunitx} % jeżeli potrzebne jednostki
\sisetup{locale = DE} % separator 1,23 (polski), można zmienić na locale=PL gdy dostępne
\usepackage{bm}       % pogrubione symbole

\usepackage[hidelinks]{hyperref}
\usepackage[nameinlink,capitalise,noabbrev]{cleveref}
\usepackage{csquotes}

\usepackage{enumitem}
\setlist{noitemsep,topsep=3pt}
\usepackage{xcolor}

\numberwithin{equation}{section}
\setcounter{secnumdepth}{3}
\setcounter{tocdepth}{2}

% --- Meta ---
\title{Zadanie 5. -- Lista 1.}
\author{Jakub Kogut}
\date{\today}

\begin{document}
\maketitle
\section{Wprowadzenie}
Do rozwiązania zadania będę wykorzystywał następujące definicje i twierdzenia podane na wykładzie.\\
Niech $L$ $L_1$ i $L_2$ będą językami nad alfabetem $\Sigma$. Wtedy:
\begin{equation}
    L_1L_2 = \left{ xy: x \in L_1, y \in L_2 \right}
\end{equation}
\begin{equation}
    L^0 = \left{ \varepsilon \right}
\end{equation}
\begin{equation}
    L^{i+1} = LL^i \quad \text{dla } i > 0
\end{equation}
\begin{equation}
    L^* = \bigcup_{i=0}^{\infty} L^i
\end{equation}
Jeżeli $r$ i $s$ są wyrażeniami regularnymi reoprezentującymi języki $L(r)$ i $L(s)$, to:
\begin{itemize}
    \item $r + s$ jest wyrażeniem regularnym reprezentującym język $L(r) \cup L(s)$,
    \item $rs$ jest wyrażeniem regularnym reprezentującym język $L(r)L(s)$,
    \item $r^*$ jest wyrażeniem regularnym reprezentującym język $L(r)^*$.
\end{itemize}

\section{Zadanie 5.}
Należy udowodnić następujące tożsamości dla wyrażeń regularnych $r$, $s$ i $t$, przy czym $r=s$ oznacza, że $L(r) = L(s)$.
\begin{enumerate}
    \item
        Kożystając z łączności sumy zbiorów mamy:
        \begin{equation}
            \begin{aligned}
                (r+s)+t &= (L(r)\cup L(s))\cup L(t) \\
                        &= L(r)\cup\bigl(L(s)\cup L(t)\bigr) \\
                        &= r + (s+t)
            \end{aligned}
        \end{equation}
    \item Z definicji konkatenacji oraz łączności konunkcji:
        \begin{equation}
            \begin{aligned}
                (rs)t &= (L(r)L(s))L(t) \\
                        &= \left\{ xyz: xy \in L(r)L(s), z \in L(t) \right\} \\
                        &= \left\{ xyz: \left( x \in L(r), y \in L(s) \right) \wedge \left( z \in L(t) \right) \right\} \\
                        &= \left\{ xyz: x \in L(r), \left( y \in L(s), z \in L(t) \right) \right\} \\
                        &= \left\{ xyz: x \in L(r), yz \in L(s)L(t) \right\} \\
                        &= L(r)(L(s)L(t)) \\
                        &= r(st)
            \end{aligned}
        \end{equation}
    \item Kożystając z rozdzielności konkatenacji względem alternatywy otrzymujemy:
        \begin{equation}
            \begin{aligned}
                (r+s) t &= (L(r)\cup L(s))\,L(t) \\
                        &= \left\{ xy: x \in L(r)\cup L(s), y \in L(t) \right\} \\
                        &= \left\{ xy: \left( x \in L(r), y \in L(t)\right) \wedge \left( x \in L(s), y \in L(t)\right) \right\} \\
                        &= \left\{ xy: x \in L(r), y \in L(t)\right\} \cup \left\{ xy: x \in L(s), y \in L(t)\right\} \\
                        &= L(r)L(t) \cup L(s)L(t) \\
                        &= rt + st
            \end{aligned}
        \end{equation}
    \item Analogicznie do poprzedniego:
        \begin{equation}
            \begin{aligned}
                r(s +t ) &= L(r)(L(s)\cup L(t)) \\
                        &= \left\{ xy: x \in L(r), y \in L(s)\cup L(t) \right\} \\
                        &= \left\{ xy: \left( x \in L(r), y \in L(s)\right) \wedge \left( x \in L(r), y \in L(t)\right) \right\} \\
                        &= \left\{ xy: x \in L(r), y \in L(s)\right\} \cup \left{ xy: x \in L(r), y \in L(t)\right\} \\
                        &= L(r)L(s) \cup L(r)L(t) \\
                        &= rs + rt
            \end{aligned}
        \end{equation}
    \item Kożystając z definicji otoczki Kleenego:
        \begin{equation}
            \begin{aligned}
                \emptyset^* &= \bigcup_{i=0}^{\infty} \emptyset^i \\
                        &= \emptyset^0 \cup \bigcup_{i=1}^{\infty} \emptyset^i \\
                        &= \left\{ \varepsilon \right\} \cup \emptyset \\
                        &= \left\{ \varepsilon \right\} \\
                        &= \varepsilon
            \end{aligned}
        \end{equation}
    \item
        Można udowodnić tożsamość $(r^*)^* = r^*$, poprzez pokazanie inkluzji w obie strony. Przyjmuje, że $L(r) = L$.
        \begin{enumerate}
            \item $L^* \subseteq (L^*)^*$\\
                Wynika to wprost z definicji:
                \[
                    L^* = (L^*)^1 \subseteq (L^*)^1 \cup \bigcup_{i=1}^{\infty} (L^*)^i  = \bigcup_{i=0}(L^*)^i = (L^*)^*
                \]
            \item $(L^*)^* \subseteq L^*$\\
                Niech $w \in (L^*)^*$. Z definicji otoczki Kleenego istnieje takie $k \geq 0$ oraz $u_1, u_2, \ldots, u_k \in L^*$ takie, że
                \[
                    w = u_1 u_2 \ldots u_k
                \]
                Natomiast każde $u_i$ można rozpisać jako
                \[
                    u_i = v_{i,1} v_{i,2} \dots v_{i,n_i}
                \]
                gdzie $n_i \geq 0$ oraz $v_{i,j} \in L$ dla $1 \leq j \leq n_i$. Zatem $w$ ma postać:
                \[
                    w = (v_{1,1} v_{1,2} \dots v_{1,n_1}) (v_{2,1} v_{2,2} \dots v_{2,n_2}) \ldots (v_{k,1} v_{k,2} \dots v_{k,n_k})
                \]
                Czyli $w$ jest konkatenacją skończonej liczby słów należących do $L$, a więc $w \in L^*$. Stąd $(L^*)^* \subseteq L^*$.
        \end{enumerate}
        A skoro obie inkluzje zachodzą, to $L^* = (L^*)^*$, czyli $(r^*)^* = r^*$.
    \item Podobnie jak powyższą tożsamość, $\big(r^* s^*\big)^* = (r+s)^*$ też można udowodnić przez inkluzję:
        \begin{enumerate}
            \item $(R^*S^*)^* \subseteq (R+S)^*$\\
                Niech $w \in (R^*S^*)^*$. Z definicji otoczki Kleenego istnieje $k \ge 0$ oraz $u_1,\dots,u_k \in R^*S^*$ takie, że
                \[
                    w = u_1 u_2 \dots u_k.
                \]
                Każdy $u_i$ ma postać $x_i y_i$ z $x_i \in R^*$ oraz $y_i \in S^*$. Dalej, z definicji:
                \[
                    x_i = r_{i,1} r_{i,2} \dots r_{i,p_i}\quad (p_i\ge 0,\ r_{i,j}\in R),
                \]
                \[
                    y_i = s_{i,1} s_{i,2} \dots s_{i,q_i}\quad (q_i\ge 0,\ s_{i,j}\in S).
                \]
                Zatem $w$ jest skończoną konkatenacją słów, z których każde należy do $R$ albo do $S$, czyli $w \in (R+S)^*$. Stąd $(R^*S^*)^* \subseteq (R+S)^*$.

            \item $(R+S)^* \subseteq (R^*S^*)^*$\\
                Niech $w \in (R+S)^*$. Wówczas istnieje $m \ge 0$ oraz $t_1,\dots,t_m \in R\cup S$ takie, że
                \[
                    w = t_1 t_2 \dots t_m.
                \]
                Zgrupujmy maksymalne kolejne bloki elementów z $R$ oraz z $S$. Otrzymujemy rozkład
                \[
                    w = (R\cdots R)(S\cdots S)(R\cdots R)\cdots(S\cdots S),
                \]
                gdzie każdy blok $R\cdots R$ należy do $R^*$, a każdy blok $S\cdots S$ do $S^*$. Zatem $w$ jest konkatenacją skończonej liczby elementów z $R^*S^*$, czyli $w \in (R^*S^*)^*$. Stąd $(R+S)^* \subseteq (R^*S^*)^*$.
        \end{enumerate}
        Z obu inkluzji otrzymujemy równość $\big(r^* s^*\big)^* = (r+s)^*$.
\end{enumerate}
\end{document}

